\documentclass[12pt]{report}
\usepackage[left=2cm,right=2.6cm,top=2.6cm,bottom=3cm,a4paper]{geometry}
\renewcommand{\baselinestretch}{1.3}

\usepackage[utf8]{inputenc}
\usepackage{fancyhdr}
\usepackage{amsmath, amssymb, latexsym, amsfonts, amsthm}

%% theorem specification
\theoremstyle{definition}
\newtheorem{problem}{Problem}[chapter]

%% advanced tabular
\usepackage{tabularx}
\newcolumntype{Y}{>{\centering\arraybackslash}X}

%% fraction adjustment

%%%%%%%%%%%%%%
\begin{document}
\begin{titlepage}
\centering
{\scshape\LARGE Solution to  \\
An Introduction to Population Genetics Theory \par}
\vspace{1.5cm}
{\large Author: \\
    James F. Crow \\
    Motoo Kimura \par}
\vspace{1.5cm}
Solution by \\
Hanbin Lee

\end{titlepage}

\tableofcontents

    \chapter{Models of Population Growth}

    \begin{problem}
        In a population with discrete generations and with fitness $w$, how many generations are required to double the population number?

        \begin{proof}
            By the definition of fitness in discrete generations, $w$ is given by $w = \frac{N_{i+1}}{N_i}$.
            This gives $N_{i} = w^{i} N_0$. 
            Since $N_i = 2N_0$ by the given condition, $i = \log{2}/\log{w}$.
        \end{proof}
    \end{problem}

    \begin{problem}
        How long is required for the population to double with model 2?
        
        \begin{proof}
            By the definition of fitness in continuous generations, $w$ is given by $w = \frac{1}{N} \frac{dN}{dt}$.
            This gives $\frac{1}{N}dN = wdt$ followed by $\int_{N_0}^{N} \frac{1}{N}dN = \int_{t_0}^{t} w dt$. 
            Therefore, $\log2 = \log{\frac{N}{N_0}} = w(t - t_0)$.
        \end{proof}
    \end{problem}

    \begin{problem}
        A population under model 3 has reached age stability. 
        How long, in units of $\lambda$, will be required for the population to double? 
        What is the effective generation length, defined as the unit that will give the same answer as problem 1?

        \begin{proof}
            Assume that, like the textbook, every individual lives for 5 years.

            Let $n_t = (n_{0t}, n_{1t}, \ldots, n_{4t})^T$, $(b_0, b_1, \ldots, b_4)$ and $(p_0, \ldots, p_3)$ denote the number of individuals in each age, reproduction rates of each age and probability of survival of each age respectively.
            Then the following matrix equation holds:
            \begin{equation*}
                n_t = 
                \begin{pmatrix}
                    n_{0,t}\\
                    n_{1,t}\\
                    n_{2,t}\\
                    n_{3,t}\\
                    n_{4,t}\\
                \end{pmatrix} =  
                \begin{pmatrix}
                    b_0 & b_1 & b_2 & b_3 & b_4 \\
                    p_0 & 0 & 0 & 0 & 0\\
                    0 & p_1 & 0 & 0 & 0\\
                    0 & 0 & p_2 & 0 & 0\\
                    0 & 0 & 0 & p_3 & 0\\
                \end{pmatrix}
                \begin{pmatrix}
                    n_{0,t-1}\\
                    n_{1,t-1}\\
                    n_{2,t-1}\\
                    n_{3,t-1}\\
                    n_{4,t-1}\\
                \end{pmatrix} 
                = A \cdot n_{t-1}
            \end{equation*}

            Then, by induction, we obtain $n_t = A^{t}n_0$.
            To compute the power of the matrix $A$, we compute the characteristic polynomial and assume that it has at least one zero:
            \begin{equation*}
                \det{(A-\lambda I)} = \lambda^5 -b_0\lambda^4 -p_0b_1\lambda^3 - p_0p_1b_2\lambda^2 - p_0p_1p_2b_3\lambda - p_0p_1p_2p_3b_4 
            \end{equation*}
            
            Under suitable conditions, the polynomial above has a single positive largest(in terms of absolute value) zero $\lambda^*$ and a corresponding eigenvector $l^*$.

            For a large $t$, $({l^*})^t$ dominates all other power of eigenvalues, so $M^t n_0$, which is expressed as a linear combination of power of eigenvalues, can be approximated by
            \begin{equation*}
                M^{t} n_0 \approx C M^{t} l = C {(\lambda^*)}^t l
            \end{equation*}
            where $C$ is a constant determined by the initial condition.

            $N_t$, the size of the population at time $t$, is given as the sum of entries of $n_t$.
            Therefore, $N_t = n_0 + n_1 + \ldots + n_4 = C' (\lambda^*)^t$ by the above equation.
            Hence, using $N_t = 2N_0$, we get $t = \log{2}/\log{\lambda}$.
        \end{proof}
    \end{problem}

    \begin{problem}
        Suppose you know the age-specific death rates (the probability that an individual of age x will die during the next time unit). What is the life expectancy, that is, the mean length of life? What is the median length of life?

        \begin{proof}
            Let $\{p_i\}_{i \in \mathbb{Z}_{\geq 0}}$ be the probability that an individual will survive during age $i$.
            Then the probability of survival until age t is given as the product of $p_i$ from $0$ to $t-1$.
            Thus, the expectation is given by
            \begin{equation*}
                E(X) = \sum_{k=0}^{\infty} \left( k \cdot \prod_{i=0}^{k}p_i \right)
            \end{equation*}

            The median value can also be computed in a similar manner using formulas from basic probability theory.
        \end{proof}
    \end{problem}


        \begin{problem}
            Show that equation $1.6.8a$ is correct for any number of strains.
            \begin{proof}
                Suppose that there are $k$ strains and $n_1, \ldots, n_k$ individuals for each strain. The Malthusian parameters of each strain is given as $r_1, \ldots, r_k$.

                Now we use the same method in the textbook.
                \begin{equation*}
                    \begin{split}
                        \frac{d \ln(p_1/(1-p_1))}{dt}
                        &= \frac{d \ln(n_1/(N-n_1)}{dt}  \\
                        &= \frac{d n_1}{dt} - \frac{d (N-n_1)}{dt} \\
                        &= \frac{d \ln{n_1}}{n_1 dt} - \frac{d \ln(N-n_1)}{(N-n_1) dt} \\
                        &= r_1 - \bar{r}_1
                    \end{split}
                \end{equation*}
                Here, $\bar{r}_i$ denotes the mean parameter except for the $i$-th strain. 

                Notice also that
                \begin{equation*}
                    \begin{split}
                        \frac{d \ln(p_1/(1-p_1))}{dt}
                        &= \frac{\ln{p_1}}{dt} - \frac{\ln{1-p_1}}{dt} \\
                        &= \frac{d p_1}{p_1 dt} - \frac{d (1-p_1)}{(1-p_1)dt} \\
                        & = \frac{dp_1}{p_1(1-p_1)dt}
                    \end{split}
                \end{equation*}

                Putting these two equations together gives
                \begin{equation*}
                    \begin{split}
                        \frac{dp_1}{dt} 
                        &= (r_1-\bar{r}_1)p_1(1-p_1) \\
                        &= ((N-n_1)r_1 - (n_2r_2 + \ldots + n_kr_k)) \cdot p_1 \cdot
                        \left(\frac{1}{N-n_1}\right) \cdot (1-p_1) \\
                        &= (Nr_1 - (n_1r_1 + \ldots + n_kr_k)) \cdot p_1 \cdot
                        \left(\frac{1}{N-n_1}\right) \cdot 
                        \left(\frac{N-n_1}{N}\right) \\
                        &= N(r_1 - \bar{r}) \cdot p_1 \cdot
                        \left(\frac{1}{N-n_1}\right) \cdot 
                        \left(\frac{N-n_1}{N}\right) \\
                        &= p_1(r-\bar{r})
                    \end{split}
                \end{equation*}
            \end{proof}


        \end{problem}

        \begin{problem}
            What are the median and mean length of life under model 2, expressed in terms of the death rate, d?
            \begin{proof}
                Suppose that $N_0$ individuals were born at a given time $t_0$. 
                Then the following equation holds:
                \begin{equation*}
                    \frac{dN_t}{dt} = -dN_t
                \end{equation*}
                The solution to this differential equation is 
                \begin{equation*}
                    N = N_0 e^{-dt}
                \end{equation*}
                Thus, the number of dead individuals are
                \begin{equation*}
                    N_0 - N = N_0(1-e^{-dt})
                \end{equation*}
                Therefore, the cumulative distribution function of survival time $t$ is given as
                \begin{equation*}
                    F(t) = 1- e^{-dt}
                \end{equation*}
                The resulting probability density function $f(t)$ is
                \begin{equation*}
                    f(t) = de^{-dt}
                \end{equation*}
                Hence, the expectation can be computed.
                \begin{equation*}
                    \int_{0}^{\infty} {t \cdot de^{-dt} dt} = \frac{1}{d}
                \end{equation*}
            \end{proof}
        \end{problem}
    
        \begin{problem}
            Show that the time required to change the number from $N_0$ to $N_t$ in a logistically growing population exceeds that in an unregulated population with the same intrinsic rate of increase by $\log[(K-N_0)/(K-N_t)]/r$.
            \begin{proof}
                By equation (1.6.3) given in the book,
                \begin{equation*}
                    t_l = \frac{1}{r} \log \frac{N_t(K-N_0)}{(K-N_t)N_0}
                \end{equation*}
                for logistically a growing population.

                For a exponentially growing populations,
                \begin{equation*}
                    t_e = \frac{1}{r} \log \frac{N_t}{N_0}
                \end{equation*}
                Substracting $t_e$ from $t_l$, we have
                \begin{equation*}
                    t_l - t_e = \frac{1}{r} \log \frac{K-N_0}{K-N_t}
                \end{equation*}
            \end{proof}
        \end{problem}


        \begin{problem}
            Again considering a logistic population with carrying capicity $K$, what is the time required to go from a fraction $x$ to a fraction $y$ of this capacity?
            \begin{proof}
                Simply put $N_t = yK$ and $N_0 = xK$ to (1.6.3) of textbook.
                \begin{equation*}
                    \begin{split}
                        t &= \frac{1}{r} \log \frac{yK(K-xK)}{(K-yK)xK} \\
                        &= \frac{1}{r} \log \frac{y(1-x)}{(1-y)x}
                    \end{split}
                \end{equation*}
            \end{proof}
        \end{problem}

        \begin{problem}
            One bactrium which reproduces by fission and follows a logistic growth pattern is introducted into each of several ponds.
            Show that the time required to fill a pond to half its capacity is proportional to the log of the carrying capacity.
            \begin{proof}
                Simply put $N_t=(1/2)K$ to (1.6.3) of textbook.
               \begin{equation*}
                   \begin{split}
                       t &= \frac{1}{r} \log \frac{(1/2)\cdot K(K-1)}{(K-(1/2)K)\cdot 1} \\
                         &\approx \frac{1}{r} \log{K}
                   \end{split}
               \end{equation*}

            \end{proof}
        \end{problem}

        \newpage

        \chapter{Randomly Mating Populations}
        In all problems, unless the contrary is stated, assume random mating.
        \begin{problem}
            In a population there are 8 times as many heterozygotes as homozygous recessives. 
            What is the frequency of the recessive gene?
            \begin{proof}
                Let $p$ and $q$ be the frequency of each allele.
                Then, $2pq = 8 \cdot q^2 \Rightarrow p=4q$ so $q=\frac{1}{5}$.
            \end{proof}
        \end{problem}

        \begin{problem}
            Show that, for a very rare recessive gene, the proportion of heterozygous carriers is approximately twice the frequeny of the recessive gene.
            \begin{proof}
                The proportion of heterozygotes is $2pq$. The frequency of recessive gene is $q$.
                Then the raito of these two values is $\frac{2pq}{q} = 2p \approx 2$.
            \end{proof}                
        \end{problem}

        \begin{problem}
            If 16\% of the population are Rh- (dd), what fraction of the Rh+ population (DD and Dd) are homozygous?
            \begin{proof}
                We obtain $q = \sqrt{16/100} = 0.04$. 
                The value that we are computing is given by
                \begin{equation*}
                    \frac{p^2}{p^2+2pq} 
                    = \frac{p}{p+2q} 
                    = \frac{0.96}{0.96+2 \cdot 0.04}
                    \approx 0.92
                \end{equation*}
            \end{proof}
        \end{problem}

        \begin{problem}
            From the data in problem 3, what fraction of the children from a large group of families where both parents were Rh+ would be expected to be Rh+?
            \begin{proof}
                The frequency of gamete with allele $d$ is $\frac{2pq}{2pq+2pq+2p^2} = \frac{pq}{2pq+p^2} \approx 0.038$. 
                Then the frequency of Rh+ progeny is $1-0.038^2 = 0.996$.
            \end{proof}
        \end{problem}

        \begin{problem}
            Show that if the A and B antigens of the ABO blood group system were caused by two dominant genes, independently inherited, the product of the frequency of A and B should equal the product of O and AB.
            \begin{proof}
                Let $a_1, a_2$ be the frequency of allele $A, a$ and $b_1, b_2$ be the frequency of allele $B, b$.
                Then the frequencies of $A$ and $B$ are $(a_1^2+2a_1a_2) \cdot b_2^2$ and $(b_1^2+2b_1b_2) \cdot a_2^2$.
                The frequencies of $O$ and $AB$ are $a_2^2 \cdot b_2^2$ and $(a_1^2+2a_1a_2) \cdot (b_1^2+2b_1b_2)$.
                Therefore, the products of these values are equal.
            \end{proof}
        \end{problem}

        \begin{problem}
            What is the maximum proportion of heterozygotes with two alleles? With three alleles? With $n$ alleles?
            \begin{proof}
                
                \begin{description}
                    \item[i) Biallelic]
                This is a simple example of the inequality of arthmetic and geometric means.
                Under $p+q=1$, the maximum value of $2pq$ is $1/4$ since $2 \sqrt{pq} \leq p+q$.

                    \item[ii) Triallelic]
                We need to compute the maximum value of $1-(p^2+q^2+r^2)$ under the constraint $p+q+r=1$. 
                Let $f(p,q,r) = 1-(p^2+q^2+r^2)$ and $g(p,q,r)=p+q+r$.
                Then we have to attain the maximum of $f(p,q,r)$ under $g(p,q,r)=1$.
                We proceed with the Lagrange multiplier criterion. 
                \begin{equation*}
                    \nabla{f} = \lambda \nabla{g} \Leftrightarrow
                    (-2) \cdot (p,q,r) = \lambda \cdot (1,1,1)
                \end{equation*}

                Now we have $(p,q,r) = -\frac{\lambda}{2}(1,1,1)$.
                Substituting these values to $p+q+r=1$ yields $-\frac{3\lambda}{2} = 1 $ resulting $\lambda = -\frac{2}{3}$ so $p=q=r=\frac{1}{3}$.
                Thus, we have $\frac{2}{3}$ as the maximum value.

                    \item[iii) $n$-allelic]
                        The proof is identical to that of case ii) so we omit the details.
                \end{description}
            \end{proof}

        \end{problem}

        \begin{problem}
            From the data given on color blindness, what fraction of women would be of normal vision, but carrying two different color-blind factors?
            \begin{proof}
                The solution is in the textbook
            \end{proof}
        \end{problem}

        \begin{problem}
            Show that in a randomly-mating population with two allels half the heterozygotes have heteroztgous mothers.
            \begin{proof}
                \begin{equation*}
                    \begin{split}
                        P(\mathrm{\{Het Mother\}|\{Het Child\}}) &= 
                        \frac{P(\mathrm{\{Het Mother\} \cap \{Het Child\}})}{P(\mathrm{\{Het Child\}})}\\
                        &= \frac{2pq \cdot 0.5 \cdot (p+q)}{2pq}  \\
                        &= \frac{pq}{2pq} \\ 
                        &= \frac{1}{2}
                    \end{split}
                \end{equation*}
            \end{proof}
        \end{problem}

        \begin{problem}
            Here are some hypothetical data on the frequencies of the ABO blood groups:
            \begin{center}
                \begin{tabular}{ccccccc}
                    \hline
                    Genotype: & OO & OA & AA & OB & BB & AB \\
                    Frequency: & .40 & .30 & .08 & .12 & .04 & .06\\
                    \hline
                \end{tabular}
            \end{center}
            What would be the frequencies be next generation if mating were at random?
            \begin{proof}
                Let frequency of A, B, O alleles be $p,q,r$.
                Their values are 
                $p = 0.5 \cdot (0.08 \cdot 2 + 0.30 + 0.06) = 0.26$,
                $q = 0.5 \cdot (0.04 \cdot 2 + 0.12 + 0.06) = 0.13$,
                $r = 0.5 \cdot (0.40 \cdot 2 + 0.30 + 0.12) = 0.61$.

                Under random mating, Hardy-Weinberg equilibrium is reached in a single generation.
                Therefore, the frequencies are:
                \begin{center}
                    \begin{tabular}{ccccccc}
                        \hline
                        Genotype: & OO & OA & AA & OB & BB & AB \\
                        Frequency: & .37 & .32 & .07 & .16 & .02 & .07\\
                        \hline
                    \end{tabular}
                \end{center}
                by $(p+q+r)^2 = p^2 + q^2 + r^2 + 2pq + 2qr + 2rp$.
            \end{proof}
        \end{problem}

        \begin{problem}
            Letting $p,q$ and $r$ stand for the frequencies of the A, B and O blood group alleles, what is the probability that two persons chosen at random have the same blood group?
            \begin{proof}
                Simply,
                \begin{equation*}
                    (p^2)^2 + (q^2)^2 + (r^2)^2 + (2pq)^2 + (2qr)^2 + (2rp)^2
                \end{equation*}
                by independence.
            \end{proof}
        \end{problem}

        \begin{problem}
            An outrageously careless hospital give transfusions at random.
            What proportion would be mismatches?
            (To refresh your memory, group O can give to anyone, A to A or AB, B to B or AB and AB only to AB.)
            \begin{proof}
                AB patients can accept any: $1$.
                A can accept O and A: $r^2+p^2+2rp$.
                B can accept O and B: $r^2+q^2+2rq$.
                O can accept only O: $r^2$.
                Sum up these values proportional to the frequency of the blood group.
            \end{proof}
        \end{problem}
        
        \begin{problem}
            Show that if $p$ is the frequency of a recessive allele, the average proportion of recessive children when one parent is of the dominant phenotype and the other of the recessive phenotype is $p/(1+p)$, and when both parents are of the dominant phenotype is $[p/(1+p)]^2$.
            \begin{proof}
                Note that the parent with dominant phenotype can be either heterozygous or homozygous.
                \begin{description}
                    \item[i) Dominant-Recessive]{
                        \begin{equation*}
                            \frac{2pq \cdot (1/2) + q^2 \cdot 0}
                            {2pq+q^2} 
                            = \frac{pq}{1-p^2}
                            = \frac{p}{1+p}
                        \end{equation*}
                    }
                    \item[ii) Dominant-Dominant]{
                        \begin{equation*}
                            \begin{split}
                            \frac{2pq \cdot 2pq \cdot (1/4)}{2pq\cdot 2pq + q^2 \cdot q^2 + 2\cdot q^2 \cdot 2pq} 
                            &= \frac{p^2q^2}{4p^2q^2+q^4+4p^3q} \\
                            &= \frac{p^2q^2}{(q^2+2pq)^2} \\
                            &= \left(\frac{pq}{1-p^2}\right)^2 \\
                            &= \left( \frac{p}{1+p} \right)^2
                        \end{split}
                        \end{equation*}
                    }
                \end{description}
                
            \end{proof}
        \end{problem}

        \begin{problem}
            The two ratios, $p/(1+p)$ and $[p/(1+p)]^2$, are sometimes called Snyder's ratios and the fact that one is the square of the other is sometimes used as a test for recessive inheritance.
            Does this discriminate between a trait caused by a single pair of recessive genes and one caused by simultaneous homozygosity for several recessive genes?
        \end{problem}

        \begin{problem}
            Does the answer to problem 13 depend on whether the genes are independent or not? Must they be in gamtiec phase equilibrium?
        \end{problem}

        \begin{problem}
            A. G. Searle (\textit{J. Genet.} 56: 1-17, 1959) reports the following frequencies of coat colors of cats in Singapore.
            The observed numbers were as follows:

            \begin{center}
                \begin{tabularx}{0.7\textwidth}{|*{6}{Y|}}
                \hline 
                \multicolumn{3}{|c|}{Females} &
                \multicolumn{2}{c|}{Males} \\
                \hline
                Dark & Calico & Yellow & Dark & Yellow \\
                \hline
                +/+ & +/y & y/y & + & y \\
                \hline
                63 & 55 & 12 & 74 & 38 \\
                \hline
            \end{tabularx}
            \end{center}
            
            Use the maximum-likelihood method to compute the gene frequency and test by Chi-square the agreement with the hypothesis of random-mating proportions.
        \end{problem}
           
\end{document}
