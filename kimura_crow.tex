\documentclass[12pt]{report}
\usepackage[left=2cm,right=2.6cm,top=2.6cm,bottom=3cm,a4paper]{geometry}
\renewcommand{\baselinestretch}{1.3}

\usepackage[utf8]{inputenc}
\usepackage{fancyhdr}
\usepackage{amsmath, amssymb, latexsym, amsfonts, amsthm}

%% theorem specification
\newtheorem{problem}{Problem}[section]
\newtheorem{solution}{Solution}



%%%%%%%%%%%%%%5
\begin{document}
\begin{titlepage}
\centering
{\scshape\LARGE Solution to  \\
An Introduction to Population Genetics Theory \par}
\vspace{1.5cm}
{\large Author: \\
    James F. Crow \\
    Motoo Kimura \par}
\vspace{1.5cm}
Solution by \\
Hanbin Lee

\end{titlepage}

\tableofcontents

\chapter{Models of Population Growth}

\begin{problem}
    In a population with discrete generations and with fitness $w$, how many generations are required to double the population number?

    \begin{proof}
        By the definition of fitness in discrete generations, $w$ is given by $w = \frac{N_{i+1}}{N_i}$.
        This gives $N_{i} = w^{i} N_0$. 
        Since $N_i = 2N_0$ by the given condition, $i = \log{2}/\log{w}$.
    \end{proof}
\end{problem}

\begin{problem}
    How long is required for the population to double with model 2?
    
    \begin{proof}
        By the definition of fitness in continuous generations, $w$ is given by $w = \frac{1}{N} \frac{dN}{dt}$.
        This gives $\frac{1}{N}dN = wdt$ followed by $\int_{N_0}^{N} \frac{1}{N}dN = \int_{t_0}^{t} w dt$. 
        Therefore, $\log2 = \log{\frac{N}{N_0}} = w(t - t_0)$.
    \end{proof}
\end{problem}

\end{document}
